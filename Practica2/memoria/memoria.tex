\documentclass[10pt,a4paper]{article}
\usepackage[utf8]{inputenc}
\usepackage[spanish]{babel}
\usepackage{amsmath}
\usepackage{amsfonts}
\usepackage{amssymb}
\usepackage{graphicx}
\usepackage[left=2cm,right=2cm,top=2cm,bottom=2cm]{geometry}
\usepackage[hidelinks]{hyperref}
\usepackage{listings}
\lstset{
    frame = single,     
	framexleftmargin = 15pt
}

\begin{document}

\begin{titlepage}
\title{\textbf{{\Huge Práctica 2 - Sistemas Legados}}}
\author{
	Pedro Allué Tamargo (758267)
	\and
	Juan José Tambo Tambo (755742)
	\and
	Jesús Villacampa Sagaste (755739)
}
\clearpage\maketitle
\thispagestyle{empty}
\tableofcontents
\end{titlepage}

\section{Esfuerzos invertidos}

\begin{itemize}
\item Pedro Allué Tamargo:
\item Juan José Tambo Tambo:
\item Jesús Villacampa Sagaste:
\end{itemize}

\section{Instalación del emulador}

Para la realización de esta práctica se ha utilizado el Sistema Operativo \emph{Ubuntu}. Para instalar el emulador \emph{x3270}\footnote{\url{http://x3270.bgp.nu/}} se ejecutarán las órdenes:

\begin{lstlisting}
sudo apt update
sudo apt -y install x3270
\end{lstlisting}

Estas instrucciones instalarán el emulador y las herramientas de \emph{scrapping} (\emph{s3270}).\\

Para conectar el \emph{scrapper} con el \emph{mainframe} se ejecutará la orden:

\begin{lstlisting}
s3270 155.210.152.51:101
\end{lstlisting}


\section{Descripción de la aplicación legada}

La aplicación legada se corresponde con una lista de tareas. El usuario podrá añadir dos tipos distintos de tareas: tareas generales y tareas específicas.\\
Esta distinción implica que las tareas específicas disponen de un campo ``nombre'' del que no disponen las generales.\\
Otro punto a tener en cuenta de la aplicación legada es que las tareas guardadas durante una ejecución no son persistentes. Es decir, las tareas no se conservan de una ejecución a otra.

\textbf{\begin{Huge}
Poner los problemas de longitud y los espacios en las tareas.
\end{Huge}}

\section{Implementación del Wrapper}

Se pide realizar una aplicación con interfaz gráfica que encapsule el acceso a la aplicación legada. Se ha elegido \emph{Java} como lenguaje para implementar este \emph{wrapper}.

\subsection{Modelo de datos}

El modelo de datos se compone de 2 clases  (Figura \textbf{PONER NUMERO}).

\textbf{PONER AQUI UML}

Las clases se componen de los mismos elementos que la aplicación legada. En el caso de \emph{GeneralTask} se compone de 2 campos: fecha y descripción. En el caso de \emph{SpecificTask} se compone de 3 campos: fecha, nombre y descripción.

\subsection{Interfaz gráfica de usuario}

\subsection{Screen scrapping}




\end{document}